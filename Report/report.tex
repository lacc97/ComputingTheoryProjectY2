\documentclass[11pt]{article}

%packages
\usepackage[british]{babel}
\usepackage{mathtools}              % AMS Math Package
\usepackage{amsthm}                 % Theorem Formatting
\usepackage{physics}
\usepackage{graphicx}               % Allows for eps images
% \usepackage{multicol}             % Allows for multiple columns
% \usepackage{sectsty}                % custom sectionals
\usepackage{titlesec}
\usepackage[small]{caption}         % small captions
\usepackage{newtxtext,newtxmath}    % font change
% \usepackage{lmodern}
\usepackage[T1]{fontenc}
\usepackage[utf8]{inputenc}
\usepackage{csquotes}
\usepackage{enumerate}              % custom enumerations
\usepackage{url}                    % nice url fonts
\usepackage[a4paper,top=2.54cm,bottom=2.54cm,left=3.18cm,right=3.18cm]{geometry}
\usepackage[separate-uncertainty=true]{siunitx}
\usepackage{float}
\usepackage[style=ieee,backend=biber]{biblatex}
\usepackage{tikz}
\usepackage{parskip}
\usepackage{booktabs}
\usepackage{multirow}
%end packages

\usepackage{subcaption}

%commands
\renewcommand{\labelenumi}{(\alph{enumi})}
\newtheorem{prop}{Proposition}
\newtheorem{thm}{Theorem}[section]
\newtheorem{lem}[thm]{Lemma}
\theoremstyle{definition}
\newtheorem{dfn}{Definition}
\theoremstyle{remark}
\newtheorem*{rmk}{Remark}

\bibliography{report}

\titlelabel{\thetitle.\quad}

\titleformat*{\section}{\normalfont\bfseries}
\titlespacing*{\section}{0pt}{1.1\parskip}{.33\parskip}
\titleformat*{\subsection}{\normalfont\bfseries}
\titlespacing*{\subsection}{0pt}{1.1\parskip}{.33\parskip}
\titleformat*{\subsubsection}{\normalfont\bfseries}
\titlespacing*{\subsubsection}{0pt}{1.1\parskip}{.33\parskip}

\let\oldvec=\vec
\renewcommand{\vec}[1]{\vb{#1}}
\newcommand{\uvec}[1]{\vu{#1}}

%cover page command
\newcommand{\makecover}[5]{
\thispagestyle{empty}
\setcounter{page}{0}
\begin{center}\LARGE{\bf #1}\vskip 24pt \normalsize{#2}\hspace*{\fill}\\
#3\vskip 12pt School of Physics and Astronomy\\The University of Manchester\vskip 12pt Theory Computing Project Report\vskip 12pt #4\end{center}\vskip 54pt

\section*{Abstract}
#5
\newpage}
%end of coverpage command

\pagestyle{plain}%page numbers in footer

\begin{document}
    \makecover
    {Neutron stars}
    {Tomoi Goto and Luis Caceres Cueva}
    {10139083 and 9916879}
    {May 2019}
    {In this project, the effects of the internal structure of the neutron stars on their maximum radius and mass were examined by solving the Tolman-Oppenheimer-Volkoff equation with appropriate equation of states using 4th order Runge-Kutta method. For the equation of state, pure Fermi gas consisting of proton and neutron was used, and different strength of nuclear interactions were considered by varying the nuclear incompressibility. [Results comes here].}
    
    \section{Introduction}
        [What's neutron star]. [Why solve eos? (Because can test nuclear physics by comparing to actual observation)].
    
    \section{Theory}
        \subsection{Structure equation}
            For a fluid in hydrostatic equilibrium, we have \[\rho\vec{f}=\grad{p},\] where $\rho$ is the density, $\vec{f}$ is the force per unit mass and $p$ is the pressure. Hence, when the force is Newtonian gravity, \[\dv{p}{r}=-\frac{G\rho M}{r^{2}},\] where $r$ is the distance from the center of the neutron star, $p=p(r)$ is the pressure at radius $r$, $\rho=\rho(r)$ is the density at radius $r$, $M=M(r)$ is the mass of the inner shell of radius $r$ and $G$ is the gravitational constant. In a general relativistic treatment, some correction factors must be incorporated \parencite{silbar.reddy.2004/neutron.stars}, and hence we write Equation \ref{eq/tov.pressure},
            \begin{align}
                \label{eq/tov.pressure}\dv{p}{r}&=-\frac{G\rho M}{r^{2}}\qty[1+\frac{p}{\epsilon}]\qty[1+\frac{4\pi r^{3}p}{Mc^{2}}]\qty[1-\frac{2GM}{c^{2}r}]^{-1}\\
                \label{eq/tov.mass}\dv{M}{r}&=4\pi r^{2}\rho,
            \end{align}
            where $\epsilon=\epsilon(r)$ is the energy density at radius $r$ and $c$ is the speed of light. Equation \ref{eq/tov.mass} is straightforward from the definition of the mass of an inner spherical shell.

            
        \subsection{Fermi gas model} % TODO cite pdf and koonin
            The neutron matter inside of a neutron star can be modeled as a degenerate Fermi gas at temperature $T=\SI{0}{\kelvin}$.\footnote{This is a valid approximation because the Fermi temperature at the densities we are considering (up to \SI{e21}{\kilogram\per\meter\cubed}) is in the order of \SI{e12}{\kelvin}, which is much higher than core temperatures of very massive stars at the end of their lives ($\sim\SI{e9}{\kelvin}$).} In this model, the energy density can be written as \parencite{koonin.1986/computational.physics/rk4.white.dwarf}
            \[\epsilon=2\int_{0}^{k_{F}}E(k)g(k)\dd{k},\] where $g(k)=k^{2}/2\pi^{2}$ is the density of states function, $E(k)=\sqrt{\hbar^{2}c^{2}k^{2}+m^{2}c^{4}}$ is the relativistic energy of the state with wavenumber $k$, and $m$ is the mass of the fermion (in this case, a neutron). With the substitution $y=\hbar k/mc$ and defining a function \parencite{koonin.1986/computational.physics/rk4.white.dwarf} \[I(x)\equiv \frac{3}{8x^{3}}\qty[x(1+2x^{2})(1+x^{2})^{1/2}-\log(x+\sqrt{1+x^{2}})]\] in terms of dimensionless $x=\hbar k_{F}/mc$ to simplify the notation, the energy density can be written as \[\epsilon=\epsilon_{0}x^{3}I(x),\] where $\epsilon_{0}=m^{4}c^{5}/\pi^{2}\hbar^{3}$ is a constant.
            
            Calculating the number density is simpler, \[n=2\int_{0}^{k_{F}}g(k)\dd{k}=\frac{k_{F}^{3}}{3\pi^{2}}=n_{0}x^{3},\] where $n_{0}=m^{3}c^{3}/3\pi^{2}\hbar^{3}$ is a constant. Hence, calculating the mass density is straightforward, \[\rho=mn_{0}x^{3}=\rho_{0}x^{3},\] where $\rho_{0}=mn_{0}$ is another constant.
            
            In summary, we have the following quantities in terms of dimensionless $x$ (which in turn is a function of the Fermi wavenumber $k_{F}$):
                \begin{align}
                    x&=\frac{\hbar k_{F}}{mc}\\
                    n&=n_{0}x^{3},\\
                    \rho&=\rho_{0}x^{3},\\
                    \epsilon&=\epsilon_{0}x^{3}I(x).
                \end{align}
                
            Hence, we can calculate the pressure \[p=\qty(\pdv{U}{V})_{N}=n^{2}\qty[\pdv{n}(\frac{\epsilon}{n})]_{N}=\frac{1}{3}\epsilon_{0}x^{4}I'(x).\]

        
        \subsection{Fermi gas with nucleon interactions} % TODO cite pdf, PAL and Bludman-Dover
            In general, we can write that the energy per nucleon for symmetric matter (i.e. number of protons equal to the number of neutrons) is given by \parencite{silbar.reddy.2004/neutron.stars} \[\frac{\epsilon(n)}{n}=mc^{2}I(x)+V(u)\] where $x=\sqrt[3]{n/n_{0}}$ is as in the previous section, $V(u)$ is the nucleon interaction potential written in terms of the dimensionless quantity $u=n/n_{1}$, where $m$ is the mass of a nucleon (i.e. we ignore the small difference between the proton and neutron mass) and $n_{1}$ is the saturation density (i.e. the density at which the potential $V$ is the minimum and is equal to the binding energy per nucleon of elemental matter). We also make the assumption that this potential applies to all neutron stars. % TODO explain this
            
            For small $x$, we can approximate $I(x)$ as \[I(x)\approx1+\frac{3}{10}x^{2}.\] Writing $x^{3}=\eta u$, where $\eta=n_{1}/n_{0}$ is the ratio between constants, we have \begin{equation}\label{eq/symmetric.energy}\frac{\epsilon(n)}{n}=mc^{2}\qty(1+\frac{3\eta^{2/3}}{10}u^{2/3})+V(u).\end{equation} % TODO explainl why we can use non relativistic approximation
            
            We can parametrise the equations for antisymmetric matter using a parameter $\alpha$ such that $n_{p}=(1-\alpha)n/2$ and $n_{n}=(1+\alpha)n/2$ where $n_{p}$ and $n_{n}$ are the proton density and neutron density, respectively \parencite{silbar.reddy.2004/neutron.stars}. Hence, we can write an approximation for the energy per nucleon of antisymmetric matter \parencite{prakash.ainsworth.lattimer.1988/eos}, \begin{equation}\label{eq/antisymmetric.energy}E(n,\alpha)=E(n,0)+\alpha^{2}S(n)\end{equation} where $E(n,0)$ is the energy per nucleon of symmetric matter given by Equation \ref{eq/symmetric.energy} and $S(n)$ is the symmetry breaking energy. Following Prakash et al. \cite{prakash.ainsworth.lattimer.1988/eos}, we use \begin{equation}\label{eq/symmetry.breaking}S(n)=E_{N}(2^{2/3}-1)\qty(u^{2/3}-F(u))+S_{0}F(u)\end{equation} where $u=n/n_{1}$ as before, $S_{0}=\SI{30}{\mega\electronvolt}$ \parencite{silbar.reddy.2004/neutron.stars} is the bulk symmetry energy parameter, $F(u)$ is an arbitrary function such that $F(0)=0$ and $F(1)=1$ and \[E_{N}=\frac{3}{10}mc^{2}\eta^{2/3}\] is the kinetic energy per nucleon at equilibrium density $n=n_{1}$. Thus we can write the energy per nucleon for pure neutron matter ($\alpha=1$) as \[E(n,1)=\frac{\epsilon}{n}=mc^{2}+E_{N}u^{2/3}+V(u)+E_{N}(2^{2/3}-1)\qty(u^{2/3}-F(u))+S_{0}F(u).\] Now we can find an expression for the pressure, \[p=n^{2}\pdv{n}\qty(\frac{\epsilon}{n})=E_{N}n_{1}u^{2}\qty[\frac{2}{3}u^{-1/3}+\frac{V'(u)}{E_{N}}+(2^{2/3}-1)\qty(\frac{2}{3}u^{-1/3}-F'(u))+\frac{S_{0}}{E_{N}}F'(u)].\] To simplify calculations we define \[J(u)\equiv u^{2}\qty[\frac{2}{3}u^{-1/3}+\frac{V'(u)}{E_{N}}+(2^{2/3}-1)\qty(\frac{2}{3}u^{-1/3}-F'(u))+\frac{S_{0}}{E_{N}}F'(u)]\] such that we can write \[p=E_{N}n_{1}J(u).\]
            
            \subsubsection{The symmetric nucleon potential}
                So far, we have an expression for the equation of state as the pressure in terms of a function $J(u)$ of density $u=n/n1$, but this function in turn is written in terms of the nucleon interaction potential $V(u)$, for which we do not yet have an expression.
                
                In this work, we use the potential proposed by Bludman and Dover \cite{bludman.dover.1980/extrapolation.skyrme.eos}. This form of the nucleon interaction potential is based on the empirical Skyrme interaction and is modified to account for higher densities than atomic nuclear densities while ensuring that the sound speed in such matter is below the speed of light \cite{bludman.dover.1980/extrapolation.skyrme.eos}. This potential can be written as \parencite{silbar.reddy.2004/neutron.stars} \[V(u)=\frac{A}{2}u+\frac{Bu^{\sigma}}{(1+\sigma)(1+C_{1}u^{\sigma-1})},\] where $A$, $B$, $C_{1}$ and $\sigma$ are constants chosen to fit the parameters of nuclear matter at saturation nuclear densities \parencite{bludman.dover.1980/extrapolation.skyrme.eos}.
                
                Hence, using Equation \ref{eq/symmetric.energy} to express the energy per nucleon, we must have:
                \begin{enumerate}
                    \item the energy per nucleon has a minimum at the saturation density $n=n_{1}$, \[\qty(\dv{u}\qty[\frac{\epsilon(n)}{n}])_{u=1}=0;\]
                    \item the value of the energy per nucleon at the minimum must be equal to the binding energy per nucleon $E_{B}=-\SI{16}{\mega\electronvolt}$ \parencite{silbar.reddy.2004/neutron.stars}, \[\frac{\epsilon(n_{1})}{n_{1}}-mc^{2}=E_{B};\]
                    \item the value of the compressibility at the saturation density must be $K_{0}$ which has a value between \SI{200}{\mega\electronvolt} and \SI{400}{\mega\electronvolt} \parencite{silbar.reddy.2004/neutron.stars},\footnote{In this work, we will fit the parameters of the nucleon interaction potential for different values of $K_{0}$.} \[K(n_{1})=9\dv{p}{n}=K_{0};\]
                    \item the value of $C_{1}$ is chosen such that the speed of sound in the matter is lower than the speed of light \parencite{bludman.dover.1980/extrapolation.skyrme.eos}, \[\qty(\frac{c_{s}}{c})^{2}=\pdv{p}{\epsilon}<1.\]
                \end{enumerate}

                The potential, although empirical, can be interpreted physically. Given that $A$ is negative, the term in $u$ represents the attractive strong force interaction of nearest neighbour nucleons. The other term instead represents the multiple body repulsive interactions with neighbours \parencite{bludman.dover.1980/extrapolation.skyrme.eos} (such as Coulombic interactions).
            
        \subsection{Observation method}
            [Methods to determine the mass and radius of neutron stars].
    
    \section{Methodology}
        [Python, C++, used RK4 with parameter of ].
    
    \section{Results and discussion}
        [Show Radius vs Pressure/Mass diagram]. [Show Mass vs Radius diagram]. []
        
        [Say the effect of nuclear interaction on the mass and radius]. [Compare the result with other models using diagram]. [Pick few models and explain a bit deeper, why they have their distinct shapes].
        
        \subsection{Error}
            [Explain and justify sources of error].
    
    \section{Conclusion}
    
    \printbibliography
\end{document}
