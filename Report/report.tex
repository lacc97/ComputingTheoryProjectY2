\documentclass[11pt]{article}

%packages
\usepackage[british]{babel}
\usepackage{mathtools}              % AMS Math Package
\usepackage{amsthm}                 % Theorem Formatting
\usepackage{physics}
\usepackage{graphicx}               % Allows for eps images
% \usepackage{multicol}             % Allows for multiple columns
% \usepackage{sectsty}                % custom sectionals
\usepackage{titlesec}
\usepackage[small]{caption}         % small captions
\usepackage{newtxtext,newtxmath}    % font change
% \usepackage{lmodern}
\usepackage[T1]{fontenc}
\usepackage[utf8]{inputenc}
\usepackage{csquotes}
\usepackage{enumerate}              % custom enumerations
\usepackage{url}                    % nice url fonts
\usepackage[a4paper,top=2.54cm,bottom=2.54cm,left=3.18cm,right=3.18cm]{geometry}
\usepackage[separate-uncertainty=true]{siunitx}
\usepackage{float}
\usepackage[style=ieee,backend=biber]{biblatex}
\usepackage{tikz}
\usepackage{parskip}
\usepackage{booktabs}
\usepackage{multirow}
%end packages

\usepackage{subcaption}

%commands
\renewcommand{\labelenumi}{(\alph{enumi})}
\newtheorem{prop}{Proposition}
\newtheorem{thm}{Theorem}[section]
\newtheorem{lem}[thm]{Lemma}
\theoremstyle{definition}
\newtheorem{dfn}{Definition}
\theoremstyle{remark}
\newtheorem*{rmk}{Remark}

\bibliography{report}

\titlelabel{\thetitle.\quad}

\titleformat*{\section}{\normalfont\bfseries}
\titlespacing*{\section}{0pt}{1.1\parskip}{.33\parskip}
\titleformat*{\subsection}{\normalfont\bfseries}
\titlespacing*{\subsection}{0pt}{1.1\parskip}{.33\parskip}
\titleformat*{\subsubsection}{\normalfont\bfseries}
\titlespacing*{\subsubsection}{0pt}{1.1\parskip}{.33\parskip}

\let\oldvec=\vec
\renewcommand{\vec}[1]{\vb{#1}}
\newcommand{\uvec}[1]{\vu{#1}}

%cover page command
\newcommand{\makecover}[5]{
\thispagestyle{empty}
\setcounter{page}{0}
\begin{center}\LARGE{\bf #1}\vskip 24pt \normalsize{#2}\hspace*{\fill}\\
#3\vskip 12pt School of Physics and Astronomy\\The University of Manchester\vskip 12pt Theory Computing Project Report\vskip 12pt #4\end{center}\vskip 54pt

\section*{Abstract}
#5
\newpage}
%end of coverpage command

\pagestyle{plain}%page numbers in footer

\begin{document}
    \makecover
    {Neutron stars}
    {Tomoi Goto and Luis Caceres Cueva}
    {10139083 and 9916879}
    {May 2019}
    {In this project, the effects of the internal structure of the neutron stars on their maximum radius and mass were examined by solving the Tolman-Oppenheimer-Volkoff equation with appropriate equation of states using 4th order Runge-Kutta method. For the equation of state, pure Fermi gas consisting of proton and neutron was used, and different strength of nuclear interactions were considered by varying the nuclear incompressibility. [Results comes here].}
    
    \section{Introduction}
        [What's neutron star]. [Why solve eos? (Because can test nuclear physics by comparing to actual observation)].
    
    \section{Theory}
        \subsection{Structure equation}
            For a fluid in hydrostatic equilibrium, we have \[\rho\vec{f}=\grad{p},\] where $\rho$ is the density, $\vec{f}$ is the force per unit mass and $p$ is the pressure. Hence, when the force is Newtonian gravity, \[\dv{p}{r}=-\frac{G\rho M}{r^{2}},\] where $r$ is the distance from the center of the neutron star, $p=p(r)$ is the pressure at radius $r$, $\rho=\rho(r)$ is the density at radius $r$, $M=M(r)$ is the mass of the inner shell of radius $r$ and $G$ is the gravitational constant. In a general relativistic treatment, this equation is corrected by some factors, and hence we obtain Equation \ref{eq/tov.pressure},
            \begin{align}
                \label{eq/tov.pressure}\dv{p}{r}&=-\frac{G\rho M}{r^{2}}\qty[1+\frac{p}{\epsilon}]\qty[1+\frac{4\pi r^{3}p}{Mc^{2}}]\qty[1-\frac{2GM}{c^{2}r}]^{-1}\\
                \label{eq/tov.mass}\dv{M}{r}&=4\pi r^{2}\rho,
            \end{align}
            where $\epsilon=\epsilon(r)$ is the energy density at radius $r$ and $c$ is the speed of light. Equation \ref{eq/tov.mass} is straightforward from the definition of the mass of an inner spherical shell.

            
        \subsection{Fermi gas model} % TODO cite pdf and koonin
        
            \[\epsilon=2\int_{0}^{k_{F}}E(k)g(k)\dd{k},\] where $g(k)=k^{2}/2\pi^{2}$ is the density of states function, and $E(k)=\sqrt{\hbar^{2}c^{2}k^{2}+m^{2}c^{4}}$ is the relativistic energy of the state with wavenumber $k$. With the substitution $y=\hbar k/mc$ and defining a function \[I(x)\equiv \frac{3}{8x^{3}}\qty[x(1+2x^{2})(1+x^{2})^{1/2}-\log(x+\sqrt{1+x^{2}})]\] in terms of dimensionless $x=\hbar k_{F}/mc$ to simplify the notation, the energy density can be written as \[\epsilon=\epsilon_{0}x^{3}I(x),\] where $\epsilon_{0}=m^{4}c^{5}/\pi^{2}\hbar^{3}$ is a constant.
            
            Calculating the number density is simpler, \[n=2\int_{0}^{k_{F}}g(k)\dd{k}=\frac{k_{F}^{3}}{3\pi^{2}}=n_{0}x^{3},\] where $n_{0}=mww^{3}c^{3}/3\pi^{2}\hbar^{3}$ is a constant. Hence, calculating the mass density is straightforward, \[\rho=mn_{0}x^{3}=\rho_{0}x^{3},\] where $\rho_{0}=mn_{0}$ is another constant.
            
            In summary, we have the following quantities in terms of dimensionless $x$ (which in turn is a function of the Fermi wavenumber $k_{F}$):
                \begin{align}
                    x&=\frac{\hbar k_{F}}{mc}\\
                    n&=n_{0}x^{3},\\
                    \rho&=\rho_{0}x^{3},\\
                    \epsilon&=\epsilon_{0}x^{3}I(x).
                \end{align}
                
            

        
        \subsection{Fermi gas with nucleon interactions} % TODO cite pdf, PAL and Bludman-Dover
            In general, we can write that the energy per nucleon for symmetric matter (i.e. number of protons equal to the number of neutrons) is given by \[\frac{\epsilon(n)}{n}=mc^{2}I(x)+V(u)\] where $x=\sqrt[3]{n/n_{0}}$ is as in the previous section, $V(u)$ is the nucleon-nucleon interaction potential written in terms of the dimensionless quantity $u=n/n_{1}$, where $m$ is the mass of a nucleon (i.e. we ignore the small difference between the proton and neutron mass), $n_{1}$ is the equilibrium density (i.e. the density at which the potential $V$ is the minimum and is equal to the binding energy per nucleon). We also make the assumption that this potential applies to all neutron stars. % TODO explain this
            
            For small $x$, we can approximate $I(x)$ as \[I(x)\approx1+\frac{3}{10}x^{2}.\] Writing $x^{3}=\eta u$, where $\eta=n_{1}/n_{0}$ is the ratio between constants, we have \[\frac{\epsilon(n)}{n}=mc^{2}(1+\frac{3\eta^{2/3}}{10}u^{2/3})+V(u).\] % TODO explainl why we can use non relativistic approximation
            
            \subsubsection{The symmetric nucleon-nucleon potential} % TODO cite PAL and Bludman-Dover
            
        \subsection{Observation method}
            [Methods to determine the mass and radius of neutron stars].
    
    \section{Methodology}
        [Python, C++, used RK4 with parameter of ].
    
    \section{Results and discussion}
        [Show Radius vs Pressure/Mass diagram]. [Show Mass vs Radius diagram]. []
        
        [Say the effect of nuclear interaction on the mass and radius]. [Compare the result with other models using diagram]. [Pick few models and explain a bit deeper, why they have their distinct shapes].
        
        \subsection{Error}
            [Explain and justify sources of error].
    
    \section{Conclusion}
    
    \printbibliography
\end{document}
