\documentclass[11pt]{article}
% Setting the margin
\usepackage[top=2.54cm, bottom=2.54cm, left=2.75cm, right=2.75cm]{geometry}
\usepackage{graphicx}
% Citation style
\usepackage{cite}
\newcommand{\citet}{\cite}
\bibliographystyle{ieeetr}
% Spacing between the lines
\linespread{1.2}
% No indentation at the start of paragraph
\setlength{\parindent}{0cm}
% Graphics
\usepackage{graphicx}
\usepackage{subcaption}
\usepackage{float}

\begin{document}


% The title page comes here
\begin{titlepage}
\begin{center}
{\Huge The effect of nuclear interaction on maximum Neutron Star mass and radius (subject to change)  }\\[0.5cm]
\textit{Luis Caceres Cueva} and \textit{Tomoi Goto}~\\[0.3cm]
\textit{student number one} and \textit{10139083}~\\[0.3cm]
School of Physics and Astronomy~\\[0.3cm]
University of Manchester~\\[0.3cm]
Theroy computing project report~\\[0.3cm]
May 2019~\\[2cm]

\end{center}
% Abstract comes here
{\Large \textbf{Abstract}}~\\[0.3cm]
In this project, the effects of the internal structure of the neutron stars on their maximum radius and mass were examined by solving the Tolman-Oppenheimer-Volkoff equation with appropriate equation of states using 4th order Runge-Kutta method. For the equation of state, pure Fermi gas consisting of proton and neutron was used, and different strength of nuclear interactions were considered by varying the nuclear incompressibility. [Results comes here].
\end{titlepage}
\pagenumbering{gobble}
\clearpage
\pagenumbering{arabic}
\setcounter{page}{2}

\newpage

% Introduction
\section{Introduction}
[What's neutron star]. 
[Why solve eos? (Because can test nuclear physics by comparing to actual observation)].
\label{intro}


% Theory
\section{Theory}

% subsection 1
\subsection{Structure Equation }
[Structure equation]. [General relativistic modification]. 
% subsection 2
\subsection{Equation of state}
[Equation of state]. [Fermi gas]. [Nuclear interaction].
% subsection 3
\subsection{Observation method}
[Methods to determine the mass and radius of neutron stars].

% Apparatus
\section{Method}
[Python, C++, used RK4 with parameter of ~].


% Result
\section{Results}
[Show Radius vs Pressure/Mass diagram]. [Show Mass vs Radius diagram]. []

% Discussion
\section{Discussion}
[Say the effect of nuclear interaction on the mass and radius]. [Compare the result with other models using diagram]. [Pick few models and explain a bit deeper, why they have their distinct shapes].

% Conclusion
\section{Conclusion}

% Bibliography
%\bibliography{reference}



\end{document}
