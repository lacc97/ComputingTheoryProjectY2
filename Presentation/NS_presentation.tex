\documentclass[hperref={pdfpagelabels=false}]{beamer}

\usepackage{hyperref}
\usepackage[utf8]{inputenc}
\usetheme{default}
\usepackage{physics}
\usepackage{caption}
\captionsetup{font=scriptsize,labelfont=scriptsize}

\begin{document}

\title
{Computing project: Neutron star}

\subtitle{Neutron star modelled as pure Fermi gas with nucleon interaction}

\author
{Luis.~Caceres.~Cueva\inst{1} \and Tomoi.~Goto\inst{1}}

\institute
{
    \inst{1}
    School of Phsyics and Astronomy\\
    The University of Manchester
}
\date
{May 8, 2019}

\begin{frame}
 \titlepage
\end{frame}

\begin{frame}
 \frametitle{Structure equations}
 Newtonian structure equation is
\[\dv{p}{r}=-\frac{G\rho M}{r^{2}}\]
\[\dv{M}{r}=4\pi r^{2}\rho.\]
General Relativistic structure equation is
\[\dv{p}{r}=-\frac{G\rho M}{r^{2}}\qty[1+\frac{p}{\epsilon}]\qty[1+\frac{4\pi r^{3}p}{Mc^{2}}]\qty[1-\frac{2GM}{c^{2}r}]^{-1}\]
\[\dv{M}{r}=4\pi r^{2}\rho.\]
\end{frame}

\begin{frame}
 \frametitle{Equation of state}
 Energy per particle is
\[\frac{\epsilon}{n}=mc^{2}+E_{N}u^{2/3}+V(u)+E_{N}(2^{2/3}-1)\qty(u^{2/3}-F(u))+S_{0}F(u),\]
 where the contribution from nuclear interaction is
 \[V(u)=\frac{A}{2}u+\frac{Bu^{\sigma}}{(1+\sigma)(1+C_{1}u^{\sigma-1})},\]
\[x=\frac{\hbar k_{F}}{mc},\]
 \[n=n_{0}x^{3},\]
 \[u=n/n_{1},\]
 and
 \[E_{N}=\frac{3}{10}mc^{2}\eta^{2/3}\]
 \[\eta=n_{1}/n_{0}.\]
\end{frame}

\begin{frame}
\frametitle{Equation of state}
 For expression in symmetry energy,
 \[F(u)=\sqrt{u}\]
 was chosen.\\
 Pressure is then 
 \[p=\qty(\pdv{U}{V})_{N}=n^{2}\qty[\pdv{n}(\frac{\epsilon}{n})]_{N}\]
 \[=E_{N}n_{1}u^{2}\qty[\frac{2}{3}u^{-1/3}+\frac{V'(u)}{E_{N}}+(2^{2/3}-1)\qty(\frac{2}{3}u^{-1/3}-F'(u))+\frac{S_{0}}{E_{N}}F'(u)],\]
 and the mass density is
 \[\rho=mn_{0}x^{3}.\]
\end{frame}


\begin{frame}
 \frametitle{Our results and comparison with other EOSs}
 \begin{columns}
  \begin{column}{0.48\textwidth}
  \begin{figure}
    \includegraphics[scale=0.3]{eos_compare_our_model.pdf}
    \caption{Mass vs radius plots of Fermi gas with Newtonian structure, General Relativistic structure, and with differing strength of nucleon interactions.}\label{fig/ourmodel}
  \end{figure}
  \end{column}
  \begin{column}{0.48\textwidth}
  \begin{figure}
    \includegraphics[scale=0.3]{eos_compare_paper.pdf}
    \caption{Mass vs radius plots comparing our model with nuclear interactions to other models in literatures.}
  \end{figure}
  \end{column}
 \end{columns}
\end{frame}

\begin{frame}
 \frametitle{Comparison with observatinos}
 \begin{columns}
  \begin{column}{0.48\textwidth}
  \begin{figure}
    \includegraphics[scale=0.3]{eos_compare_obsv1_GW.pdf}
    \caption{Comparison between our models with observations from quiescent low mass X-ray binaries and gravitational wave event.}
  \end{figure}
  \end{column}
  \begin{column}{0.48\textwidth}
  \begin{figure}
    \includegraphics[scale=0.3]{eos_compare_obsv2_GW.pdf}
    \caption{Comparison between our models with observations from thermonuclear bursts of LMXBs and gravitational wave event.}
  \end{figure}
  \end{column}
 \end{columns}
\end{frame}



\end{document}
